% Für Arbeiten auf Englisch
\documentclass{article}

% Für Arbeiten auf Deutsch
\usepackage[utf8]{inputenc}

\usepackage{graphicx}
\usepackage{amsmath}
\usepackage{amssymb}
\usepackage{color}
\renewcommand*{\thefootnote}{\fnsymbol{footnote}}
\pagestyle{plain}

\begin{document}

\section*{Aufgabe 1}
	\begin{enumerate}
		\item [a)]
		Unter der Annahme, dass beide Spieler ihren Gewinn maximieren wollen, würde
		ich den höchst möglichen Betrag x wählen. Da beide Spieler mehr oder weniger
		den gleichen Betrag bekommen, der das Minimum von beiden genannten Beträgen ist,
		sollten beide ziemlich hoch sein. Einen niedrigeren Betrag zu wählen, nur um
		2 Euro mehr zu bekommen macht keinen Sinn, wenn man damit rechnen muss, dass der
		andere auch die Strategie des Unterbietens wählt. Denn dann müsste man iterativ
		den Verlust minimieren versuchen und würde schließlich doch bei $<10$ Euro landen.
		
		Demnach würde ich folgende zwei Möglichkeiten der nennbaren Betragsbereiche
		unterscheiden:
		
		1. Möglichkeit:\\ 
		Der Betrag den der Versicherungsvertreter ohne nachzufragen akzeptiert ist in [2, 101],
		nur der geschätzte Betrag ist in [2, 100].
		
		Persönlich würde ich eine 1 Googol bieten in der Hoffnung, dass der andere auch
		einen lächerlich hohen Betrag wählt. Denn dann erhalten wir beide einen
		lächerlich hohen Betrag was ausgesprochen cool wäre. Dann gibt es
		nur 2 Möglichkeiten:
		1. Der andere Spieler wählt einen Betrag x, der darunter liegt, dann bekomme ich
		(x-2) Euro, mei schade. Das liegt aber vermutlich immer noch ziemlich nahe an
		100 Euro es sei denn der andere Spieler ist doof. Oder aber
		2. Er hat einen lächerlich höheren Betrag gewählt, dann bekommen wir beide
		immernoch etwa 1 Googol Euro. Awesome!
		
		2. Möglichkeit:\\
		Der Betrag, den die Spieler nennen dürfen ist in [2, 100], so würde ich 100 Euro
		wählen.
		Wollte man einen höheren Gewinn erzielen, hätte man nur
		die Möglichkeit 99 Euro zu bieten und (!) der andere Spieler müsste 100 Euro wählen.
		Dann würde der andere Spieler 97 Euro erhalten. Auch noch ein akzeptabler Betrag im
		Vergleich dazu, dass einer der Spieler weniger als 97 Euro bietet und damit beide
		weniger als 100 Euro bekämen.

		\item [b)]
			$G = \langle N, (A_i)_{i \in N}, (u_i)_{i \in N} \rangle$ \\
			$N = \{ Spieler1, Spieler2 \}$\\
			$A_{Spieler1} = A_{Spieler2} = \{ \mathit{ausweichen}, \mathit{weiterfahren}\}$\\
			$u_{Spieler1} = u_{Spieler2} = ausweichen \times weiterfahren \rightarrow \mathbb{R}$
			
			\begin{tabular}{| c | c |} \hline
				& Spieler2 \\ \hline
				Spieler1 & \begin{tabular}{ c | c | c} 
					& ausweichen & weiterfahren \\ \hline
					ausweichen & $b, b$ & \textcolor{red}{$c, a$}  \\ \hline
					weiterfahren & \textcolor{red}{$a, c$} & $d, d$ \\ 
				\end{tabular} \\ \hline   
			\end{tabular}
			
			mit $a > b > c > d$\\
			
			Nash-Gleichgewicht: einer fährt weiter, andere weicht aus, da keiner den Nutzen im Nachhinein vergrößern
			kann.
		\item [c)]
			\begin{tabular}{| c | c |} \hline
				& Spieler2 \\ \hline
				Spieler1 & \begin{tabular}{ c | c | c} 
					& ausweichen & weiterfahren \\ \hline
					ausweichen & $3, 3$ & $1, 4$  \\ \hline
					weiterfahren & $4, 1$ & $0, 0$ \\ 
				\end{tabular} \\ \hline   
			\end{tabular}
			\\
			\begin{align*}
			p \cdot 3 + (1 - p) \cdot 1 &= p \cdot 4 + (1 - p) \cdot 0\\
			3p + (1 - p) &= 4p\\
			2p + 1 &= 4p\\
			1 &= 2p\\
			\frac{1}{2} &= p
			\end{align*}
			
			anlalog ergibt sich $q = \frac{1}{2}$. Beide Spieler sollten also jeweils in 50\% der Fälle ausweichen und in den anderen 50\% weiterfahren.
					
		
		
	\end{enumerate}

\end{document}
