% Für Arbeiten auf Englisch
\documentclass{article}

% Für Arbeiten auf Deutsch
\usepackage[utf8]{inputenc}

\usepackage{graphicx}
\usepackage{amsmath}
\renewcommand*{\thefootnote}{\fnsymbol{footnote}}
\pagestyle{plain}

\begin{document}

\section*{Aufgabe 3}
	\begin{enumerate}
		\item [a)]
			$G = \langle N, (A_i)_{i \in N}, (u_i)_{i \in N} \rangle$ \\
			$N = \{ a, b, c \}$\\
			$A_a = A_b = A_c = \{ \mathit{cachen}, \mathit{holen}\}$\\
			
			$u_i = \footnote{Da die Kosten minimiert werden sollen.} \frac{1}{c_i} = 
				\begin{cases}
					1 & \textit{$A_i$ cacht Datei}\\
					0 & \textit{Kein erreichbarer Agent chacht}\\
					\frac{1} {|SP(i,j)| \cdot d_i} & sonst
				\end{cases}
			$
		\item [b)]
			Nur $b$ cacht.
		\item [c)]
			\begin{tabular}{| c | c |} \hline
				 & a \\ \hline
				b & \begin{tabular}{ c | c | c} 
						& cacht & holt \\ \hline
						cacht & $1, 1$ & $1, 2$  \\ \hline
						holt & $2, 1$ & $0, 0$ \\ 
					\end{tabular} \\ \hline   
			\end{tabular}
		\item [d)]
		\begin{tabular}{| c | c | c |}
			\hline 
			Iteration & N & S \\ \hline
			init & $\{a,b,c,d,e,f\}$ & $\o$  \\ \hline
			1 & $\{b\}$ & $\{c,e\}$  \\ \hline
			2\footnotemark & $\{b,c\}$ oder $\{b,e\}$ & $\{e\}$ oder $\{c\}$  \\ \hline
			3 & $\{b,c,e\}$ & $\o$ \\
			\hline 
		\end{tabular}
		\footnotetext{2 Möglichkeiten, da $d_c = d_e = 2$.}\\\\
			Die Agenten $b, c$ und $e$ cachen. Am Ende des Alorithmus' gilt $c_{i \leftarrow j} \le 1$ für alle Agenten $A_i$, da diese entweder entfernt werden (falls $c_{i \leftarrow j} < 1$) oder die Datei cachen ($c_{i \leftarrow j} = 1$). Alle Agenten die nicht cachen könnten ihre Kosten also lediglich auf 1 erhöhen. Entscheidet sich ein cachender Agent die Datei zu holen, so sind seine Kosten größer als 1, da dies das Kriterium für die Aufnahme in die Menge der cachenden Agenten war. Durch einseitiges Abweichen kann kein Agent seine Situation verbessern, es handelt sich also um ein Nash-Equilibrium.
	\end{enumerate}


\end{document}
