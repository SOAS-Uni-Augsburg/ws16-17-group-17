% Für Arbeiten auf Englisch
\documentclass{article}

% Für Arbeiten auf Deutsch
\usepackage[utf8]{inputenc}

\usepackage{graphicx}
\usepackage{amsmath}
\renewcommand*{\thefootnote}{\fnsymbol{footnote}}
\pagestyle{plain}

\begin{document}

\section*{Aufgabe 3}
	\begin{enumerate}
		\item [a)]
			$G = \langle N, (A_i)_{i \in N}, (u_i)_{i \in N} \rangle$ \\
			$N = \{ a, b, c \}$\\
			$A_a = A_b = A_c = \{ \mathit{cachen}, \mathit{holen}\}$\\
			
			$u_i = - c_i \footnote{Da die Kosten minimiert werden sollen.} = 
				\begin{cases}
					- 1 & \textit{$A_i$ cacht Datei}\\
					- |SP(i,j)| \cdot d_i & sonst
				\end{cases}
			$
		\item [b)]
			Nur $b$ cacht.
		\item [c)]
			\begin{tabular}{| c | c |} \hline
				 & a \\ \hline
				b & \begin{tabular}{ c | c | c} 
						& cacht & holt \\ \hline
						cacht & $-1, -1$ & $-1, -0.5$  \\ \hline
						holt & $-0.5, -1$ & $\infty, \infty$ \\ 
					\end{tabular} \\ \hline   
			\end{tabular}
		\item [d)]
			Agent $b$ hat die maximale Lesehäufigkeit $d_b=3$ und cacht.
			Agenten $a$, $c$, $d$ und $f$ cachen nicht, da für sie
			$c_{i \rightarrow b} < 1$ ist. Agent $e$ cacht. \\
			ToDo Erzeugt dieser Algorithmus stets ein Nash-Gleichgewicht?
	\end{enumerate}


\end{document}
