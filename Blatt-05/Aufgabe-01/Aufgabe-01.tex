%
% Created based on
% https://en.wikibooks.org/wiki/LaTeX/Presentations
%
\documentclass{beamer}

\usepackage[T1]{fontenc}
\usepackage[utf8x]{inputenc}
\usepackage{fourier}

\usepackage{amsmath}
\usepackage{amsfonts}

% Valid themes are:
%   Antibes, Bergen, Berkeley, Berlin, Copenhagen, Darmstadt, Dresden, Frankfurt
%   Goettingen, Hannover, Ilmenau, JuanLesPins, Luebeck, Madrid, Malmoe, Marburg
%   Montpellier, PaloAlto, Pittsburgh, Rochester, Singapore, Szeged, Warsaw
%   boxes, default, CambridgeUS
\usetheme{Madrid}

% Valid colorthemes are:
%   default, albatross, beaver, beetle, crane, dolphin, dove, fly, lily, orchid
%   rose, seagull, seahorse, whale, wolverine
\usecolortheme{crane}

\title[SOAS]{Selbstorganisierende, adaptive Systeme}
\subtitle{Übungsblatt 5}
\author[Ferdinand, Mikhail, Stefan]{ %
Ferdinand Dürlich, \\
Mikhail Kreymerman, \\
Stefan Büttner
}
\institute[ISSE]{ %
Institut for Software \& Systems Engineering\\
Universität Augsburg
}
\date{Fr., 25.11.2016}

\newcommand{\E}{\mathbb{E}}



\begin{document}

\frame{\titlepage}

\begin{frame}
\frametitle{Frame Title}
\framesubtitle{Frame Subtitle}

\begin{itemize}
    \item Some item
    \item another item
\end{itemize}

\begin{block}{block title}
    asdf
\end{block}
\begin{align*}
    \mathbb{P}(a|s) = \prod_{i \in N} s_i(a_i)
\end{align*}
\end{frame}

\begin{frame}
\frametitle{Aufgabe 1}
\framesubtitle{c)}

\begin{columns}
\column{0.25\textwidth}
\begin{tabular}{r|c|c|}
    & l & r \\
    \hline
l & $ 1,-1$ & $-1, 1$ \\
r & $-1, 1$ & $ 1,-1$ \\
    \hline
\end{tabular}

\column{0.75\textwidth}
$s_1(l) = p$, $s_2(l) = q$
\end{columns}

\begin{block}{$\E u_1(l, s_2) = \E u_1(r, s_2)$ \hfill Sp. 1 ist indifferent gegenüber Sp. 2}
\begin{align*}
    & & s_2(l)u_1(l,l) + s_2(r)u_1(l,r) & = s_2(l)u_1(r,l) + s_2(r)u_1(r,r) & \\
\Leftrightarrow &
    & q - (1-q) & = -q + (1-q) &
\Leftrightarrow
    q = \frac{1}{2}
\end{align*}
\end{block}

\begin{block}{$\E u_2(s_1, l) = \E u_2(s_1, r)$ \hfill Sp. 2 ist indifferent gegenüber Sp. 1}
\begin{align*}
    & & s_1(l)u_2(l,l) + s_1(r)u_2(l,r) & = s_1(l)u_2(r,l) + s_1(r)u_2(r,r) & \\
\Leftrightarrow &
    & -p + (1-p) & = p - (1-p) &
\Leftrightarrow
    p = \frac{1}{2}
\end{align*}
\end{block}
\end{frame}

\begin{frame}
\frametitle{Aufgabe 1}
\framesubtitle{d)}

\begin{columns}[c]
\column{0.25\textwidth}
\begin{tabular}{r|c|c|}
    & l & r \\
    \hline
l & $x, 2$ & $0, 0$ \\
r & $0, 0$ & $2, 2$ \\
    \hline
\end{tabular}

\column{0.75\textwidth}
Nash-Gleichgewicht $(s_1, s_2)$ mit $s_1(l) = p$, $s_2(l) = q$.\\
$\Rightarrow s_1(r) = 1 - p$, $s_2(r) = 1 - q$. \\
Gesucht $p(x), q(x)$.
\end{columns}

\begin{block}{$\E u_1(l, s_2) = \E u_1(r, s_2)$ \hfill Sp. 1 ist indifferent gegenüber Sp. 2}
\begin{align*}
s_2(l)u_1(l,l) + s_2(r)u_1(l,r) = s_2(l)u_1(r,l) + s_2(r)u_1(r,r)
& \Leftrightarrow & 
qx & = (1-q)2 \\
& \Rightarrow &
q & = \frac{2}{x+2}
\end{align*}
\end{block}

\begin{block}{$\E u_2(s_1, l) = \E u_2(s_1, r)$ \hfill Sp. 2 ist indifferent gegenüber Sp. 1}
\begin{align*}
s_1(l)u_2(l,l) + s_1(r)u_2(l,r) = s_1(l)u_2(r,l) + s_1(r)u_2(r,r)
& \Leftrightarrow &
2p& = 2(1-p) \\
& \Rightarrow &
p & = \frac{1}{2}
\end{align*}
\end{block}

\end{frame}

\end{document}
