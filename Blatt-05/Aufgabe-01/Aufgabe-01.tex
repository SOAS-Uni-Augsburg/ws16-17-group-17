%
% Created based on
% https://en.wikibooks.org/wiki/LaTeX/Presentations
%
\documentclass{beamer}

\usepackage[T1]{fontenc}
\usepackage[utf8x]{inputenc}
\usepackage{fourier}

\usepackage{amsmath}
\usepackage{amsfonts}
\usepackage{color}

\usepackage{cancel}

% Valid themes are:
%   Antibes, Bergen, Berkeley, Berlin, Copenhagen, Darmstadt, Dresden, Frankfurt
%   Goettingen, Hannover, Ilmenau, JuanLesPins, Luebeck, Madrid, Malmoe, Marburg
%   Montpellier, PaloAlto, Pittsburgh, Rochester, Singapore, Szeged, Warsaw
%   boxes, default, CambridgeUS
\usetheme{Madrid}

% Valid colorthemes are:
%   default, albatross, beaver, beetle, crane, dolphin, dove, fly, lily, orchid
%   rose, seagull, seahorse, whale, wolverine
\usecolortheme{crane}

\title[SOAS]{Selbstorganisierende, adaptive Systeme}
\subtitle{Übungsblatt 5}
\author[Ferdinand, Mikhail, Stefan]{ %
Ferdinand Dürlich, \\
Mikhail Kreymerman, \\
Stefan Büttner
}
\institute[ISSE]{ %
Institut for Software \& Systems Engineering\\
Universität Augsburg
}
\date{Fr., 25.11.2016}

\newcommand{\E}{\mathbb{E}}



\begin{document}

\frame{\titlepage}

\begin{frame}
\frametitle{Frame Title}
\framesubtitle{Frame Subtitle}

\begin{itemize}
    \item Some item
    \item another item
\end{itemize}

\begin{block}{block title}
    asdf
\end{block}
\begin{align*}
    \mathbb{P}(a|s) = \prod_{i \in N} s_i(a_i)
\end{align*}
\end{frame}

\begin{frame}
	\frametitle{Aufgabe 1}
	\framesubtitle{a)}
	Die iterative Elimination strikt dominanter Strategien kann unabhängig von der Reihenfolge
	der eliminierten Strategien ausgeführt werden. Daraus können sich algorithmische Vorteile
	ergeben. Allerdings gilt dies im Allgemeinen nicht für schwach dominierte Aktionen (also reine Strategien). Zeigen Sie anhand des folgenden Beispiels, dass die Reihenfolge der Elimination Auswirkungen auf den Ausgang eines Spiels haben kann.
	
	\begin{block}{Schwach dominante Aktion}
		Eine Aktion $a_i$ ist eine \textit{schwach dominante Aktion}, wenn für alle $\mathbf{a}_{-i} \in \mathcal{A}_{-i}$ und alle $a_i^{'} \in A_i$ gilt:\\
		\centering
		$u_i(\mathbf{a}_{-i}, a_i) \ge u_i(\mathbf{a}_{-i}, a_i^{'})$
	\end{block}
\end{frame}	

\begin{frame}
	\frametitle{Aufgabe 1}
	\framesubtitle{a)}
	\begin{block}{Ein möglicher Weg}	
		\centering
		\begin{tabular}{r|c|c|}
			& l & r \\
			\hline
			t & $\textcolor{green}{2}, 1$ & $\textcolor{green}{0}, 0$ \\
			m & $\textcolor{green}{2}, 1$ & $\textcolor{green}{1}, 1$ \\
			b & $0, 0$ & $1, 1$ \\
			\hline
		\end{tabular}
		$\rightarrow$
		\begin{tabular}{r|c|c|}
			& l & r \\
			\hline
			m & $2, \textcolor{blue}{1}$ & $1, \textcolor{blue}{1}$ \\
			b & $0, \textcolor{blue}{0}$ & $1, \textcolor{blue}{1}$ \\
			\hline
		\end{tabular}
		$\rightarrow$
		\begin{tabular}{r|c|}
			& r \\
			\hline
			m & $\textcolor{red}{1}, 1$ \\
			b &  $\textcolor{red}{1}, 1$ \\
			\hline
		\end{tabular}\\
		$\rightarrow$
		\begin{tabular}{r|c|}
			& r \\
			\hline
			m & $1, 1$ \\
			\hline
		\end{tabular}
		oder
		\begin{tabular}{r|c|}
			& r \\
			\hline
			b &  $1, 1$ \\
			\hline
		\end{tabular}	
	\end{block}
	\begin{itemize}
		\item Im dritten Zustand ist $m$ schwach dominant gegenüber $b$ und umgekehrt. Es kann also wahlweise $m$ oder $b$ eliminiert werden.
		\item Die iterative Elimination kann in diesem Beispiel, bei entsprechend gewählten Eliminationsschritten, zu allen Zuständen außer $\langle b, l \rangle$ und $\langle t, r \rangle$ führen.
	\end{itemize}
\end{frame}

\begin{frame}
	\frametitle{Aufgabe 1}
	\framesubtitle{b)}

	\begin{block}{Dominanzlösbarkeit}
		Ein Spiel ist \textit{dominanzlösbar}, falls die iterative Elimination exakt einen Ausgang (ein Aktionsprofil) liefert.
		Dieses ist zugleich das einzige Nash-Gleichgewicht.
	\end{block}
	\centering
	Ist folgendes Spiel dominanzlösbar?

	\begin{tabular}{r|c|c|c|}
		& $l$ & $m$ & $r$ \\
		\hline
		$u$ & $3, 8$ & $2, 0$ & $1, 2$ \\
		$d$ & $0, 0$ & $1, 7$ & $8, 2$ \\
		\hline
	\end{tabular}

\end{frame}

% Folie mit Nash-Gleichgewicht kann weggelassen werden
\begin{frame}
	\frametitle{Aufgabe 1}
	\framesubtitle{b) Nash-Gleichgewicht}

	\centering
	\begin{tabular}{r|c|c|c|}
		& l & m & r \\
		\hline
		u & $\textcolor{orange}{3}, \textcolor{red}{8}$ & $\textcolor{orange}{2}, 0$ & $1, 2$ \\
		d & $0, 0$ & $1, \textcolor{red}{7}$ & $\textcolor{orange}{8}, 2$ \\
		\hline
	\end{tabular}

	\begin{itemize}
		\item Naiver Algorithmus zur Suche von Nash-Gleichgewichten zeigt, dass es sich beim Aktionsprofil $(u, l)$ um ein Nash-Gleichgewicht handelt.
	\end{itemize}

\end{frame}

\begin{frame}
	\frametitle{Aufgabe 1}
	\framesubtitle{b) Iterative Elimination}

	% TODO: evtl. Unterscheidung nach reinen/gemischten Strategien
	% Reine Strategien: Eine strikt dominierte Aktion: es gibt eine Aktion, die diese Aktion strikt dominiert
	% Gemischte Strategien: Strikt dominierte Strategie: es gibt eine Strategie, die diese Strategie strikt dominiert

	% Definitionen:
	% strikt dominierte Aktion Folien S. 6 (nicht das gleiche wie strikt dominante Aktion Folien S. 19)
	\begin{block}{Strikt dominierte Strategie}
		% eigentlich Strikt dominierte Aktion, steht aber so in
		% http://gki.informatik.uni-freiburg.de/teaching/ss09/gametheory/spieltheorie.pdf
		% TODO: auch strikt dominierte Strategie genauer definieren
		Eine Aktion $a_i \in A_i$ heißt \textit{strikt dominiert}, falls es eine Aktion $a^+_i \in A_i$ gibt, so dass für
		alle $\mathbf{a}_{-i} \in \mathcal{A}_{-i}$ gilt:\\
		\centering
		$u_i(\mathbf{a}_{-i}, a^+_i) > u_i(\mathbf{a}_{-i}, a_i)$
	\end{block}
	Es ist nicht rational, strikt dominierte Strategien zu spielen.


	% http://gki.informatik.uni-freiburg.de/teaching/ss09/gametheory/spieltheorie.pdf 2.1.1
	\begin{block}{Iterative Elimination strikt dominierter Strategien}
		\begin{itemize}
			\item Streiche die Strategien, die strikt dominiert sind,\\ solange welche da sind.
			\item Bleibt nur ein Aktionsprofil Übrig, ist das die Lösung.
		\end{itemize}
	\end{block}

\end{frame}

\begin{frame}
	\frametitle{Aufgabe 1}
	\framesubtitle{b)}

	\centering
	\begin{tabular}{r|c|c|c|}
		& $l$ & $m$ & $r$ \\
		\hline
		$u$ & $3, 8$ & $2, 0$ & $1, 2$ \\
		$d$ & $0, 0$ & $1, 7$ & $8, 2$ \\
		\hline
	\end{tabular}

	\begin{block}{Reine Strategie}
		Spieler wählen genau eine Aktion.
	\end{block}

	\begin{itemize}
		\item Es gibt keine strikt dominierte reine Strategie (= Aktion) 
		\item Iterative Elimination  geht nicht weiter und liefert keine eindeutige Lösung\\
		$\Rightarrow$ Das Spiel ist für reine Strategien nicht dominanzlösbar
	\end{itemize}

\end{frame}

\begin{frame}
	\frametitle{Aufgabe 1}
	\framesubtitle{b)}

	\centering
	\begin{tabular}{r|c|c|c|}
		& $l$ & $m$ & $r$ \\
		\hline
		$u$ & $3, 8$ & $2, 0$ & $1, 2$ \\
		$d$ & $0, 0$ & $1, 7$ & $8, 2$ \\
		\hline
	\end{tabular}

	\begin{block}{Gemischte Strategie}
		Spieler ziehen die Aktion nach einer Wahrscheinlichkeitsverteilung.
	\end{block}

	\begin{itemize}
		\item Durch geeignete Randomisierung erwarteten Nutzen maximieren
	\end{itemize}

	\begin{tabular}{r|c|c|}
		& $l \leftarrow \frac{1}{2}, m \leftarrow \frac{1}{2} $& $r$ \\
		\hline
		$u$ & $2.5, \textcolor{red}{4}$ & $1, \textcolor{red}{2}$ \\
		$d$ & $0.5, \textcolor{red}{3.5}$ & $8, \textcolor{red}{2}$ \\
		\hline
	\end{tabular}

	\begin{itemize}
		\item In der Matrix stehen Erwartungswerte
		\item Gemischte Strategie dominiert die reine Strategie $r$ strikt.\\
		$\Rightarrow$ $r$ kann eliminiert werden
	\end{itemize}

	% TODO: auf der Nächsten Folie r eliminieren und weiter machen
	% TODO: Ein Nash-Gleichgewicht in gemischten Strategien ist ein Strategieprofil, siehe Folien S. 12
\end{frame}

\begin{frame}
	\frametitle{Aufgabe 1}
	\framesubtitle{b)}

	\centering
	\begin{tabular}{r|c|c|c|}
		& $l$ & $m$ & $r$ \\
		\hline
		$u$ & $3, 8$ & $2, 0$ & \cancel{$1, 2$} \\
		$d$ & $0, 0$ & $1, 7$ & \cancel{$8, 2$} \\
		\hline
	\end{tabular}
	$\rightarrow$
	\begin{tabular}{r|c|c|}
		& $l$ & $m$ \\
		\hline
		$u$ & $\textcolor{orange}{3}, 8$ & $\textcolor{orange}{2}, 0$\\
		$d$ & $\textcolor{orange}{0}, 0$ & $\textcolor{orange}{1}, 7$\\
		\hline
	\end{tabular}
	$\rightarrow$
	\begin{tabular}{r|c|c|}
		& $l$ & $m$ \\
		\hline
		$u$ & $3, 8$ & $2, 0$\\
		$d$ & \cancel{$0, 0$} & \cancel{$1, 7$}\\
		\hline
	\end{tabular}
	$\rightarrow$
	\begin{tabular}{r|c|c|}
		& $l$ & $m$ \\
		\hline
		$u$ & $3, \textcolor{red}{8}$ & $2, \textcolor{red}{0}$\\
		\hline
	\end{tabular}
	$\rightarrow$
	\begin{tabular}{r|c|c|}
		& $l$ & $m$ \\
		\hline
		$u$ & $3, 8$ & \cancel{$2, 0$}\\
		\hline
	\end{tabular}
	$\rightarrow$
	\begin{tabular}{r|c|}
		& $l$\\
		\hline
		$u$ & $3, 8$\\
		\hline
	\end{tabular}

	\begin{itemize}
		\item Iterative Elimination leifert exakt einen Ausgang (ein Aktionsprofil).\\
		% Dieses ist zugleich das einzige Nash-Gleichgewicht. TODO: evtl. Bezug zum naiven Algorithmus nehmen
		$\Rightarrow$ Das Spiel ist für gemischte Strategien dominanzlösbar
	\end{itemize}

\end{frame}

\begin{frame}
\frametitle{Aufgabe 1}
\framesubtitle{c)}

\begin{columns}
\column{0.25\textwidth}
\begin{tabular}{r|c|c|}
    & l & r \\
    \hline
l & $ 1,-1$ & $-1, 1$ \\
r & $-1, 1$ & $ 1,-1$ \\
    \hline
\end{tabular}

\column{0.75\textwidth}
$s_1(l) = p$, $s_2(l) = q$
\end{columns}

\begin{block}{$\E u_1(l, s_2) = \E u_1(r, s_2)$ \hfill Sp. 1 ist indifferent gegenüber Sp. 2}
\begin{align*}
    & & s_2(l)u_1(l,l) + s_2(r)u_1(l,r) & = s_2(l)u_1(r,l) + s_2(r)u_1(r,r) & \\
\Leftrightarrow &
    & q - (1-q) & = -q + (1-q) &
\Leftrightarrow
    q = \frac{1}{2}
\end{align*}
\end{block}

\begin{block}{$\E u_2(s_1, l) = \E u_2(s_1, r)$ \hfill Sp. 2 ist indifferent gegenüber Sp. 1}
\begin{align*}
    & & s_1(l)u_2(l,l) + s_1(r)u_2(l,r) & = s_1(l)u_2(r,l) + s_1(r)u_2(r,r) & \\
\Leftrightarrow &
    & -p + (1-p) & = p - (1-p) &
\Leftrightarrow
    p = \frac{1}{2}
\end{align*}
\end{block}
\end{frame}

\begin{frame}
\frametitle{Aufgabe 1}
\framesubtitle{d)}

\begin{columns}[c]
\column{0.25\textwidth}
\begin{tabular}{r|c|c|}
    & l & r \\
    \hline
l & $x, 2$ & $0, 0$ \\
r & $0, 0$ & $2, 2$ \\
    \hline
\end{tabular}

\column{0.75\textwidth}
Nash-Gleichgewicht $(s_1, s_2)$ mit $s_1(l) = p$, $s_2(l) = q$.\\
$\Rightarrow s_1(r) = 1 - p$, $s_2(r) = 1 - q$. \\
Gesucht $p(x), q(x)$.
\end{columns}

\begin{block}{$\E u_1(l, s_2) = \E u_1(r, s_2)$ \hfill Sp. 1 ist indifferent gegenüber Sp. 2}
\begin{align*}
s_2(l)u_1(l,l) + s_2(r)u_1(l,r) = s_2(l)u_1(r,l) + s_2(r)u_1(r,r)
& \Leftrightarrow & 
qx & = (1-q)2 \\
& \Rightarrow &
q & = \frac{2}{x+2}
\end{align*}
\end{block}

\begin{block}{$\E u_2(s_1, l) = \E u_2(s_1, r)$ \hfill Sp. 2 ist indifferent gegenüber Sp. 1}
\begin{align*}
s_1(l)u_2(l,l) + s_1(r)u_2(l,r) = s_1(l)u_2(r,l) + s_1(r)u_2(r,r)
& \Leftrightarrow &
2p& = 2(1-p) \\
& \Rightarrow &
p & = \frac{1}{2}
\end{align*}
\end{block}

\end{frame}

\end{document}
